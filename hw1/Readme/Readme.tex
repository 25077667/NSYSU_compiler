\documentclass[12pt,a4paper]{article}
\usepackage{setspace}
\usepackage{CJKutf8}
\usepackage[utf8]{inputenc}
\usepackage{verbatim}
\usepackage[hidelinks]{hyperref}
\usepackage{listings}
\usepackage{color}
\font\mytitle=cmr12 at 36pt
\definecolor{light-gray}{gray}{0.95}

\title{\mytitle Homework 1 \\ Readme of \\ Simple Java -- Scanner}
\author{Chih-Hsuan Yang\\
National Sun-Yet-San University, Taiwan \\
Instructor: Ye-In Chang
}

\begin{document}
\begin{CJK}{UTF8}{bsmi}
    \begin{spacing}{1.5}
        \maketitle
    \end{spacing}
    \newpage
    \tableofcontents
    \newpage

    % ==============================
    \section{Lex version}
    \fbox{
        \parbox{\textwidth}{
            \$ lex --version \\
            \input{|"lex --version"}
        }
    }

    \section{Operating System}
    \fbox{
        \parbox{\textwidth}{
            \$ uname -voms
            \verbatiminput{|"uname -voms"}
    }
    }

    \section{Execution manual}
    \subsection{Build}
    \paragraph{Symbol table}\mbox{(Built, but if you want to compile by yourself)}\\
    \fbox{
        \parbox{\textwidth}{
            \$ cd lib \&\& mkdir build \&\& cd build\\
            \$ cmake ..\\
            \$ make \\
            \$ mv libsymbol\_table.a ..
        }
    }
    \paragraph{Lexer}\mbox{}\\
    \fbox{
        \parbox{\textwidth}{
            \$ flex -o lex.c demo.l\\
            \$ gcc -c lex.c\\
            \$ g++ lex.o lib/libsymbol\_table.a -lfl -o my\_lex.elf
        }
    }

    \paragraph{\LaTeX}\mbox{}\\
    \fbox{
        \parbox{\textwidth}{
            \$ cd Readme\\
            \$ make
        }
    }

    \subsection{Run}
    \fbox{
        \parbox{\textwidth}{
            \$ \# Build it first\\
            \$ ./my\_lex.elf $<$ input.java
        }
    }


    \section{How to implement}
    Regex in Lex and std::unordered\_set in C++

    \section{Problems}
    \begin{enumerate}
        \item About the symbol table:
              the requirement has said:\\
              \fbox{
                  \parbox{.9\textwidth}{
                      create()     建立一個symbol table。\\
                      lookup(s)    傳回字串s的index;假如s沒找到的話,就傳回-1。\\
                      insert(s)    新增s到symbol table中,並傳回存放位置的index。\\
                      dump()       將symbol table中所有的資料印出。
                  }
              }
              \begin{enumerate}
                  \item About the lookup(s) function:\\
                        Why you indexing? And how you indexing?\\
                        The symbol table is a set mathematically. A set does not have any argument is called
                        index.\\
                        However, the requirement asked:\\
                        \fbox{
                            \parbox{.83\textwidth}{
                                lookup(s)    傳回字串s的index;假如s沒找到的話,就傳回-1。\\
                            }
                        }
                        Why?\\
                        How can I return a index of a set?\\
                        What is the index of an element in a set?
                  \item About the create() function:\\
                        This is an implementation issue.\\
                        According to the RAII pattern: In RAII, holding a resource is a
                        class invariant, and is tied to object lifetime: resource allocation
                        (or acquisition) is done during object creation (specifically
                        initialization), by the constructor, while resource deallocation
                        (release) is done during object destruction (specifically finalization),
                        by the destructor. \\
                        Therefore, request a function to construct an object, but no destruct
                        is unreasonable. It would cause the memory leak issue.

                  \item About the dump() function:\\
                        Well, this is a naming issue.\\
                        We use the word "dump" in programming, that means some error(s)
                        happened, and we would get some information about the "crash" event.
                        However, there is no "crash" happened while we were printing the
                        symbol table, right?\\
                        Hence, we shouldn't take the "dump" to be the function name.\\
                        For clean code, we should named it as "export" or "view". Which
                        give users a decision to handle the exported table, no matter they
                        want to "print", "store" etc.\\
                        Again, we shouldn't use "dump" here.
              \end{enumerate}

        \item The GNU Flex does not fully support POSIX:\\
              I tried $[$:alpha:$]$ for $[$A-Za-z$]$ in regex. However, the flex does
              not support.\\
              And I tried the dollar sign '\$' for the "End-of-line". How ever the flex
              does not support too.

        \item The test file is DOS format:\\
              Well, this is not a mistake. I just not want to precessing these formatting issue.
              Every UNIX(-like) system are using '$\backslash$n' as newline. I don't want
              to add a special for DOS format.
    \end{enumerate}

    \section{Outcome}
    \lstinputlisting[showspaces=flase,
        backgroundcolor=\color{light-gray},
        basicstyle=\scriptsize,]{outcome}

\end{CJK}
\end{document}